\documentclass[../main.tex]{subfiles}
% !TEX root= ../main.tex

\begin{document}

\subsection{Moore's Law}

\begin{itemize}
	\item \textbf{Moore's Law:} The number of transistors on a chip will double every year from 1965 through 1975.
	\item The rate has gradually slowed from doubling every year to doubling every 3 or 4 year.
\end{itemize}

\subsection{Energy and Time}

\begin{itemize}
	\item \textbf{Deaned Scaling:}
	      \begin{itemize}
		      \item Gate delay scales as \(\lambda\)
		      \item Clock frequency scales as \(\frac{1}{\lambda}\)
		      \item Power = Energy / Time
		      \item Number of devices on a chip scales as \(\lambda^{-2}\)
	      \end{itemize}
\end{itemize}

\subsection{The End? of Moore's Law}

\begin{itemize}
	\item The power wall --- chips are at the cooling limit.
	\item The atom wall --- transistors size are now a few tens of atoms.
\end{itemize}

\subsection{Why Parallelism Matter?}

\begin{itemize}
	\item Greater throughput with a huge number of simpler, lower clock frequency processors.
	\item The only way go grow performance is with more parallelism.
\end{itemize}

\end{document}
