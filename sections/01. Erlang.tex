\documentclass[../main.tex]{subfiles} 
% !TEX root= ../main.tex

\begin{document}

\begin{lstlisting}[language=erlang]
    % boolean expressions
    not true.
    true and false.
    true or true.
    true xor true.
    
    length(List). % return the length of List
    max(A,B). % return the max
    abs(A). % return the absolute value
    element(Index, Tuple). % return the element at position Index in the Tuple
    	[X * X || X <- lists:seq(1,5)]. % [1,4,9,16,25]
    
    % lists
    lists:seq(N). % return  a list with length N
    lists:sum(List). % return the sum of List
    lists:map(fun() -> ok end, List). % apply function to each element in the List
    lists:nth(N, List). % return Nth element in the List
    lists:foldl(fun(Val, Acc) -> ok end, Acc, List).
    lists:mapfoldl(fun(X, Sum) -> {2*X, X+Sum} end, 0, [1,2,3,4,5]). % {[2,4,6,8,10],15}
    
    % if-statement
    if a == b -> 1;
    b == c -> 2;
    true -> 3
    end.
    
    % switch-statement
    case Q of
    a -> 1;
    b -> 2;
    c -> 3
    end.
    
    % sending and receiving messages
    Pid ! Expr.
    receive
    Pattern1 -> Expr1;
    Pattern2 -> Expr2;
    end.
    % a new process is created- the child
    % when fun returns, the child process terminates, return value is discarded.
    spawn(fun()-> ok end)
    
    % bitwise
    2#0101 bor 2#1100 -> 2#1101. % base2
    16#a -> 10. % base16
    
    Acc ++ List. % O(|Acc|) work for the ++ operation.
\end{lstlisting}


\begin{itemize}
	\item \textbf{Referential Transparency:} a variable get a value when it is declared and that the value of the variable never changes.
	\item \textbf{Tail-recursive:} In comparison to headFactorial, tailFactorial is tail recursive because the recursive call is the very last thing that the function does.
	\item \textbf{Head-recursive:} the recursion call is made while there is still a pending operation for the current call of the function.
	\item \textbf{Tail-call elimination:} overwrite the caller’s stack frame with a new frame for caller (tail-call elimination turns tail-recursive functions into while-loops)
\end{itemize}

\end{document}
