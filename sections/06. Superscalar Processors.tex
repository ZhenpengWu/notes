\documentclass[../main.tex]{subfiles}
% !TEX root= ../main.tex

\begin{document}

\begin{itemize}
	\item \textbf{Dependencies:}
	      \begin{itemize}
		      \item Read-After-Write (RAW): \(inst_j\) reads a register or memory location written by \(inst_i\)
		      \item Write-After-Read (WAR): \(inst_j\) writes a location read by \(inst_i\)
		      \item Write-After-Write (WAW): \(inst_i\) and \(inst_j\) write the same location.
		      \item Control Dependencies: \(inst_i\) is a control flow instruction, we need to execture the correct instructions.
	      \end{itemize}
	\item \textbf{Hazards:}
	      \begin{itemize}
		      \item \textbf{Data Hazards}:
		            \begin{itemize}
			            \item an instruction read a different value than would have been read with a sequential execution of instructions
			            \item if a register or memory location is left holding a different value value than it would have had in a sequential execution
		            \end{itemize}
		      \item \textbf{Control Hazards}:
		            \begin{itemize}
			            \item an instruction is executed that would not have been executed in a sequential execution.
			            \item This is because the instruction depends on a jump or branch that hasn't finished in time.
		            \end{itemize}
	      \end{itemize}
	      \item \textbf{Handling Hazards:}
		            \begin{itemize}
			            \item Bypass, if an instruction has a result that a later instruction needs, the earlier instruction can provide that result directly without waiting to go through the register file.
			            \item Move common operations early.
			            \item If nothing else helps, stall.
		            \end{itemize}
\end{itemize}

\subsection{Register Renaming}

\begin{itemize}
	\item \textbf{False Dependencies}: WAR: if we had a fresh variable, we could have the write go to a new register and not interface with the earlier (in program order) read. Likewise for WAW.
	\item When an instruction is decoded:
	      \begin{itemize}
		      \item The logical register that it reads are bound to physical registers according to the current register mapping
		      \item A fresh register is allocated for the register it writes. The register is marked as \textbf{busy}.
	      \end{itemize}
	\item When an instruction updates its destination register, it changes it from \textbf{busy} to \textbf{ready}.
	\item An instruction can execute when all registers that it reads are ready.
\end{itemize}

\subsection{Branch Prediction / Speculative Execution}

\begin{itemize}
	\item Track statistics for branch outcomes.
	\item Speculatively execute the more likely path.
	\item Roll-back if wrong.
\end{itemize}

\subsection{Superscalar Execution}

\begin{itemize}
	\item Fetch several instructions each cycle.
	\item Decode them in parallel, and send them to issue queues for the appropriate functional unit.
	\item Handling Dependencies using \textbf{Register Renaming} and \textbf{Branch Speculation}
	\item \textbf{Multi-threading:} The features for executing multiple instruction in parallel work well for mixing instructions from several threads or processes.
\end{itemize}

\end{document}
