\documentclass[../main.tex]{subfiles}
% !TEX root= ../main.tex

\begin{document}

\subsection{The Parallel Random Access Machine(PRAM) Model}

\begin{itemize}
	\item A computer is composed of multiple processors and a \textit{shared memory}.
	\item The processor are like those from the RAM model, operate in lockstep.
	\item The memory allows each processor to perform a read or write in a single step. Multiple reads and writes can be performed in the same cycle.
\end{itemize}

\subsection{The Work-Span Model}

\begin{align*}
	T_\infty \leq T_P \leq \frac{1}{P}(T_1 - T_\infty) + T_\infty
\end{align*}

\begin{itemize}
	\item \textbf{Limitation:} Work-span ignores communication cost.
\end{itemize}

\subsection{The Candidate Type Architecture (CTA) Model}

\begin{itemize}
	\item A computer is composed of multiple processors.
	\item Each processor has \textit{local memory} that can be accessed in a single processor step.
	\item A small number of connections to a communication network.
	\item Sending \(W\) words cost \(\lambda + t_w W\), \(\lambda\) means communication delay and overhead.
	\item \textbf{Limitation:} Communication is expensive, but it doesn’t explicitly charge for bandwidth.
\end{itemize}

\subsection{The LogP Model}

\begin{itemize}
	\item \code{L}, the latency of the communication network fabric
	\item \code{o}, the overhead of a communication action
	\item \code{g}, the bandwidth of the communication network
	\item \code{P}, the number of processors
	\item \textbf{Limitation:} logP accounts for bandwidth, but doesn’t recognize that all bandwidth is not the same.
\end{itemize}

\end{document}
