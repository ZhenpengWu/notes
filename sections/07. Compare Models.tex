\documentclass[../main.tex]{subfiles}
% !TEX root= ../main.tex

\begin{document}

Whether shared memory or message passing is faster depends on the problem being solved, the quality of the implementations, and the system (s) it is running on. For example, on a single server, it will probably be easier and higher performance to use a shared memory programming environment. Across a distributed cluster, it will probably be faster to use a message passing library.

\subsection{Shared Memory}

\begin{itemize}
	\item \textbf{Advantages:}
	      \begin{itemize}
		      \item \textbf{High bandwidth:} the buses that connect the cache can be very wide, especially if the caches are on a single chip.
		      \item \textbf{Low latency:} the hardware handles moving the data -- no os calls and context-switch overheads.
		      \item One thread can pass an entire data structure to another thread just by giving a pointer.
		      \item No need to pack-up trees, graphs, or other data structure as messages and unpack them at the receiving end.
	      \end{itemize}
	\item \textbf{Disadvantages:}
	      \begin{itemize}
		      \item   {
		            Poor scaling of coherence models to large numbers of processors.
		            Shared memory \textbf{doesn't scale} as well as message passing.
		            \begin{itemize}
			            \item In a message passing machine, each CPU has its own memory, nearby and fast.
			            \item For large machines, the latency of directory accesses can severely degrade performance.
			            \item Shared memory moves the data after the cache miss, this stalls a thread.
		            \end{itemize}
		            }
		      \item It's easy to overlook synchronization (control to shared data structures). Then we get data races, corrupted data structures, and other hard-to-track--down bugs.
		      \item A defensive reaction is to wrap every shared reference with a lock, but locks are slow.
	      \end{itemize}
\end{itemize}

\subsection{Message Passing}

\begin{itemize}
	\item \textbf{Advantages:} reduce the need for synchronization construct.
	\item \textbf{Disadvantages:} Network delays.
\end{itemize}

\subsection{Superscalar}

\begin{itemize}
	\item \textbf{Advantages:}
	      \begin{itemize}
		      %   \item reduce the need for synchronization construct.
		      \item Scientific computing
		      \item Commercial computing (databases, webservers):
		            \begin{itemize}
			            \item Have large data set and  high cache miss rates
			            \item find executable instructions after cache miss
		            \end{itemize}
	      \end{itemize}
	\item \textbf{Disadvantages:}
	      \begin{itemize}
		      \item Limited instruction level parallelism.
		      \item Burning lots of power
		      \item Many operations require hardware grows quadratically with W
	      \end{itemize}
\end{itemize}

\end{document}